\documentclass[12pt]{article}
\usepackage[utf8]{inputenc}
\usepackage[T2A]{fontenc}
\usepackage[russian]{babel}
\usepackage[pdftex]{graphicx}
\usepackage{amsmath}
\usepackage{geometry} % Меняем поля страницы
\geometry{left=1.5cm}% левое поле
\geometry{right=2cm}% правое поле
\geometry{top=1cm}% верхнее поле
\geometry{bottom=2cm}% нижнее поле
\graphicspath{{.}}
\DeclareGraphicsExtensions{.pdf,.png,.jpg}

\begin{document}

\begin{titlepage}
\Large

\begin{center}
Санкт-Петербургский \\ Политехнический университет Петра Великого

\vspace{10em}

Отчет по лабораторной работе №2.2\\

\vspace{2em}

\textbf{Решение задачи Коши. Метод Рунге-Кутты.}
\end{center}

\vspace{6em}

\newbox{\lbox}
\savebox{\lbox}{\hbox{Попов Павел Сергеевич}}
\newlength{\maxl}
\setlength{\maxl}{\wd\lbox}
\hfill\parbox{14cm}{
\hspace*{5cm}\hspace*{-5cm}Студент:\hfill\hbox to\maxl{Попов Павел Сергеевич\hfill}\\
\hspace*{5cm}\hspace*{-5cm}Преподаватель:\hfill\hbox to\maxl{Курц Валентина Валерьевна}\\
\\
\hspace*{5cm}\hspace*{-5cm}Группа:\hfill\hbox to\maxl{5030102/00003}\\}

\vspace{\fill}
\begin{center}
Санкт-Петербург \\2022
\end{center}
\end{titlepage}

\section{Формулировка задачи и её формализация}
\textbf{Задача:} Требуется решить задачу Коши для ОДУ $y' = \frac{y}{x} + x \cdot cos(x)$ на промежутке $[\frac{\pi}{2}, 2\pi]$ с начальным условием $y(a) = \frac{\pi}{2}$ с помощью метода Рунге-Кутты 3 порядка с шагом $\frac{h}{3}$.
\section{Алгоритм метода и условия его применимости} 
\textbf{Алгоритм метода:} \\ 
Задача Коши в простейшем случае ставится для дифференциального уравнения первого порядка с начальным условием
$$
y^{\prime}=f(x, y) \quad x \in[a, b] \quad y(a)=y_{a}
$$
Строится сетка $\left(x_{0}=a, x_{n}=b\right)$ на отрезке $[a, b]$ (в общем случае она может быть неравномерной). Методы Рунге-Кутты строятся с наперед заданной точностью s по схеме
$$
y_{i+1}=y_{i}+h_{i} \sum_{j=1}^{s} \rho_{j} k_{j} \quad k_{j}=f\left(x_{i}+\alpha_{j} h_{i}, y_{i}+h_{i} \sum_{m=1}^{j-1} \beta_{j m} k_{m}\right)
$$
Метод Рунге-Кутты 3 порядка с шагом $\frac{h}{3}$ строится по формулам следующим образом:\\
\begin{center}
$y_{i+1}=y_{i}+\frac{h}{4}(k_1 + 3k_3)$\\
$k_{2}=f\left(x_{i+\frac{1}{3}}, y_{i}+\frac{h k_{1}}{3}\right)$\\
$k_{3}=f\left(x_{i+\frac{2}{3}}, y_{i}+\frac{2 h k_{2}}{3}\right)$, где $x_{i+\frac{(1,2)}{3}} = x_i + \frac{{(1,2)}\cdot h }{3}$

\end{center}
В задании необходимо также вычислить значение функции с заданной точностью. Необходимо применить правило Рунге. Тогда алгоритм действий для получения следующего значения функции будет такой:
\begin{flushleft}
1. Вычислить $y_{i+1}$ с помощью $y_{i}$ с шагом $h=x_{i+1}-x_{i}$\\
2. Вычислить $y_{i+1}$ с помощью $y_{i}$ с шагом $h / 2$ (от точки $x_{i}$ необходимо сделать два шага до точки $x_{i+1}$ )\\
3. Вычислить поправку $\frac{y_{i+1}^{h}-y_{i t 1}^{k / 2}}{2^{k}-1}$ (здесь $k-$ порядок метода) и сравнить ее с точностью\\
4. Если точность больше поправки, то повторить действия начиная с п.2., уменьшив шаг в 2 раза.\\
5. Если поправка меньше точности, то перейти к следующей точке.
\end{flushleft}
\textbf{Условия применимости:}\\
$f$ уд. условию Липшица;\\
$y(a) \neq 0$.
\section{Предварительный анализ задачи}
Для единственности решения задачи Коши (2) достаточно потребовать:
1. $\left|\frac{\partial f}{\partial x}\right| \leq M, \forall x \in[a, b], y-$ допустимых
2. условие Липщица по переменной $y$
$$
\left|f\left(x, y_{1}\right)-f\left(x, y_{2}\right)\right| \leq L\left|y_{1}-y_{2}\right|
$$
где $x \in[a, b], y_{1}, y_{2}-$ допустимые, $L-$ константа Липшица.

\section{Проверка условий применимости метода}
Метод Рунге-Кутты 3-го порядка обладает вычислительной устойчивостью, поэтому для корректной работы алгоритма достаточно, чтобы выполнялись условия применимости:\\
1. $\left|\frac{\partial f}{\partial x}\right|= \frac{-y}{x^2} + cos(x) - x sin(x)$ непрерывна, а значит по т. Вейерштрасса, ограничена.\\
2. Проверим выполнимость условия Липщица по $y$. Для этого воспользуемся тем, что если в области допустимых значений функция $f(x, y)$ имеет непрерывную частную производную $\frac{\partial f}{\partial y}$, то в этой области выполняется условие Липшица. Как видим, $\frac{\partial f}{\partial y}=$ $\frac{-y}{x^2} + cos(x) - x sin(x)$ непрерывна, где $x \in[a, b], y-$ допустимые.\\
\section{Тестовый пример с расчетами}
Функция $f(x,y) = \frac{y}{x} + x\cdot cos(x), n = 4, h = 1.178, \frac{h}{2} = 0.589, \varepsilon = 0.1. $\\
Шаг h:\\
$k1 = f(1.570, 1.570) = 1 $\\
$k2 = f(1.570 + \frac{1.178}{3}, 1.570 + 1.178 * 1 / 3 ) = 0.248$\\
$k3 = f(1.570 + \frac{2 * 1.178}{3}, 1.570 + \frac{2* 1.178 * 0.248}{3}) = -0.916$\\
$y_{h} = 1.570 + \frac{1.178}{4}* (1 + 3* (-0.916)) = 1.055$\\
Первый шаг $\frac{h}{2}$\\
$k1 = f(1.570, 1.570) = 1 $\\
$k2 = f(1.570 + \frac{0.589}{3}, 1.570 + 0.589 * 1 / 3 ) = 0.655$\\
$k3 = f(1.570 + \frac{2 * 0.589}{3}, 1.570 + \frac{2* 0.589 * 0.248}{3}) = 0.179$\\
$y_{\frac{h}{2}} = 1.570 + \frac{0.589}{4}* (1 + 3* (-0.916)) = 1.797$\\
Второй шаг $\frac{h}{2}$\\
$k1 = f(2.159, 1.797) = -0.367 $\\
$k2 = f(2.159 + \frac{0.589}{3}, 1.797 + 0.589 * 1 / 3 ) = -0.933$\\
$k3 = f(2.159 + \frac{2 * 0.589}{3}, 1.797 + \frac{2* 0.589 * 0.248}{3}) = -1.561$\\
$y_{\frac{h}{2}} = 1.797 + \frac{0.589}{4}* (1 + 3* (-0.916)) = 1.053$\\
Оценка по Рунге: \\
$\frac{|y_h-y_{h/2}|}{7} = 0.002$.\\
Абсолютная погрешность:\\
$|y(1.570 + 1.178) - y_{\frac{h}{2}}| = |1.0519 - 1.053| = 0.0011$\\
Заметим, что оценка Рунге позволяет найти решение с достаточной точностью.\\
За 2 шага по $\frac{h}{2}$ точность получена выше заданной.
\section{Модульная структура программы}
\begin{itemize}
    \item double f(double x, double y) - функция, для которой выполняется поиск первообразной.
    \item double sol(double x) - первообразная (решение ОДУ)
    \item std::pair<double, int> RKE(std::function<double(double, double)> f, double xprev, double yprev, double h, double eps) метод Рунге-Кутты с проверкой по правилу Рунге. В общем случае сетка неравномерная.
    \item void LocalErrPlot() - данные для графиков
    \item void Plot() - данные для графиков
    \item void EpsPlot() - данные для графиков
\end{itemize}
\section{Перечень контрольных тестов}
Необходимо построить следующие графики:
\begin{itemize}
    \item Построить график точного и численного решения, а также график ошибки.
    \item Исследовать зависимость локальной и глобальной прогрешности от h.
    \item Применить правило Рунге. Проверить достигается ли заданная точность. Исследовать влияние заданной точности на объем вычислений.
\end{itemize}

\section{Численный анализ задачи}
\begin{figure}[h!]
\center
\includegraphics[width=0.5\textwidth]{func_err.png}
\caption{График функции и ошибка в узловых точках}
Из графика видно, что при количестве узлов $n = 10$ приближение выполнено достаточно точно. Ошибка в узловых точках не превышает $\varepsilon = 1e-3$, что достаточно точно при небольшом количестве узлов.
\end{figure}
\begin{figure}[h!]
\center
\includegraphics[width=0.5\textwidth]{local_global.png}
\caption{Локальная и глобальная ошибки}
Глобальная ошибка - максимальная ошибка из всех шагов приближения. Заметим, что последующее шаг метода выполняется на основании предыдущего, значит, ошибка накапливается. По графику видно, что глобальная ошибка имеет порядок $O(h^3)$, локальная (первая точка) -$O(h^4)$, что соответствует теории. \\
\end{figure}
\begin{figure}[h!]
\center
\includegraphics[width=0.5\textwidth]{acc_err.png}
\caption{Достижение точности и вычислительные затраты (итерации)}
График ошибки лежит ниже требуемого $\varepsilon$, что значит, что метод глобальная ошибка не превышает заданной точности. Локальная ошибка лежит выше $\varepsilon$, что не нарушает теоретических утверждений.
\end{figure}

\newpage
\newpage
\section{Краткие выводы}
В данной работе я проанализировал поиск решения ОДУ с помощью метода Рунге-Кутты 3 порядка. Было установлено, что локальная ошибка имеет порядок выше глобальной на единицу, что соответствует теории, метод сходится точно.\\
При условии недостаточно точного последующего значения $y_{i+1}$ можно применить правило Рунге,уменьшив шаг в два раза для нахождения последующей координаты.\\
Количество итераций, подсчитанное по формуле количество вызовов функции $f()$, разделенное на три, растет с увеличением требуемой точности.
\end{document}
