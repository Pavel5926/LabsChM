\documentclass[12pt]{article}
\usepackage[utf8]{inputenc}
\usepackage[T2A]{fontenc}
\usepackage[russian]{babel}
\usepackage[pdftex]{graphicx}
\usepackage{amsmath}
\usepackage{geometry} % Меняем поля страницы
\geometry{left=1.5cm}% левое поле
\geometry{right=2cm}% правое поле
\geometry{top=1cm}% верхнее поле
\geometry{bottom=2cm}% нижнее поле
\graphicspath{{.}}
\DeclareGraphicsExtensions{.pdf,.png,.jpg}

\begin{document}

\begin{titlepage}
\Large

\begin{center}
Санкт-Петербургский \\ Политехнический университет Петра Великого

\vspace{10em}

Отчет по лабораторной работе №2.2\\

\vspace{2em}

\textbf{Интерполяция сплайнами.}
\end{center}

\vspace{6em}

\newbox{\lbox}
\savebox{\lbox}{\hbox{Попов Павел Сергеевич}}
\newlength{\maxl}
\setlength{\maxl}{\wd\lbox}
\hfill\parbox{14cm}{
\hspace*{5cm}\hspace*{-5cm}Студент:\hfill\hbox to\maxl{Попов Павел Сергеевич\hfill}\\
\hspace*{5cm}\hspace*{-5cm}Преподаватель:\hfill\hbox to\maxl{Курц Валентина Валерьевна}\\
\\
\hspace*{5cm}\hspace*{-5cm}Группа:\hfill\hbox to\maxl{5030102/00003}\\}

\vspace{\fill}
\begin{center}
Санкт-Петербург \\2022
\end{center}
\end{titlepage}
\section{Формулировка задачи и её формализация}
\textbf{Задача:} Дана гладкая функция $f = \sqrt{x} + cos(x)$, негладкая функция $q = \sqrt{x} + |cos(x)|$. Дан набор точек $\displaystyle \{{x_i}\}_0^n, \{{y_i}\}_0^n$ для каждой функции. Необходимо построить интерполяционный естественнный сплайн $\displaystyle S_3^1$, т.е собрать систему кубических функций, приближающих заданные функции. 
\section{Алгоритм метода и условия применимости}
Необходимо построить естественный кубический сплайн. \\
\textbf{Сетка:} \\
Для построения сплайнов воспользуемся равномерной сеткой.\\
Для задания полинома по точкам нужно построить сетку. Сетка выбрана равномерная. Первая и вторая половины точек задаётся формулой \begin{center}
$\displaystyle x_i = b - (b-a)\biggr(\frac{n-i}{n}\biggl)$
\end{center} $a, b$ - границы отрезка для интерполяции.\\
Необходимо найти 4n коэффициентов для n функций вида:
\begin{center}
$\displaystyle g_i = a_ix^3 + b_ix^2 + c_i + d , i = 1 ... n$
\end{center}
Из формулировки задачи имеется условие
\begin{equation*}
 \begin{cases}
   $g_i(x_i) = g_{i+1}(x_i)$, 
   \\
  $g_i'(x_i) = g_{i+1}'(x_i)$,
   \\
  $g_i''(x_i) = g_{i+1}''(x_i)$
 \end{cases}
\end{equation*}
непрерывности до второй производной для всех $i = 1 ... n - 1$.\\
Также имеется условие интерполирования: \begin{center}
$g_1(x_0) = y_0, g_i(x_i-1) = y_i$ для всех $i = 1 ... n - 1$.
\end{center}
Суммарно имеется $3(n-1) + n + 1 = 4n-2$ условий.\\
Добавив еще 2 условия (называемых граничными), мы сможем решить систему уравнений и найти все функции: $g_0''(a) =g_n''(b) = 0 $ (условие естественного сплайна).\\
В качестве граничных условий могут выступать значения первой либо второй производной искомой функции на границах промежутка интерполирования.\\
Замена: 
\begin{center}
    $g_i''(x_i) = M_i, h_j = x_j - x_{j-1},j =0...n, i = 1...n$
\end{center}\\
Т.к. $g$ - полином 3 степени, то $g''$ - линейная функция.\\

\begin{center}
$\displaystyle g''(x) = M_{i-1}\frac{(x_i - x)}{h_i} + M_{i}\frac{(x - x_{i-1})}{h_i}, x\in [x_{i-1},x_i] (*)$\\
\end{center}
Проинтегрировав $(*)$ два раза, получим: \begin{center}
$ \displaystyle g_i(x) =  M_{i-1}\frac{(x_i - x)^3}{6h_i} +  M_{i}\frac{(x - x_{i-1})^3}{6h_i} + C_i(x-x_i) + \overline{C_i}, i = 1...n$.\\
\end{center}
Получено уравнение для сплайна на $i$ отрезке.\\
Для того, чтобы найти $Mi$ запишем в матрицу, которая, очевидно, окажется трехдиагональной, решив её методом прогонки.\\
После этого найдём неизвестные $C_i$ и $\overline{C_i}$ по формулам:\\
\begin{center}
 $\displaystyle C_i = \frac{y_i - y_{i-1}}{h_i} - \frac{h_i}{6}(M_i - M_{i-1})$;\\
$\displaystyle \overline{C_i} = y_{i-1} - M_{i-1}\frac{h_i^2}{6}$.\\   
\end{center}
По полученным уравенениям и построим приближающие кривые.
\subsection{Условия применимости:}\\
Требуется упорядоченный набор $\{{x_i}\}_0^n, \{{y_i}\}_0^n$, также для устойчивости метода прогонки требуется диагональное преобладание в матрице.
\section{Предварительный анализ задачи}
Задана табличная функция с помощью узлов $(x_i, y_i), i = \overline{1,n}$ , потребуем выполнения условия интерполяции на каждом подотрезке $[x_i, x_{i+1}]:\phi(x_i) = y_i$, что можно записать в виде СЛАУ. Отсюда следует, что интерполяционный полином существует и единственен для каждого указанного подотрезка.\\
\section{Проверка условий применимости метода}
Условия применимости выполнены автоматически исходя из выбора стартовых точек. Также диагональное преобладание автоматически выполнено, т.к. матрицы на диагонали $M(i,i) = 2$.\\
\section{Тестовый пример с расчетами}
$\mathrm{f(x) = \sqrt{x} + cos (x)} $\\
Промежуток = $[5, 10]$, равномерная сетка: $x^h = [5, 6.25, 7.5, 8.75, 10.0]$, \\ $y^h = [2.51973, 3.49944, 3.08524, 2.1771, 2.32320],\  h_i = 1.66$ для всех $i$.\\
Будем искать сплайн в виде функции $g$, составленной из функций $g_i$, определенных на последовательно взятых частях отрезка интерполирования. \\\\
Замена. $g''(x_i) = M_i,\; i=0,\ ...,\ n$. $M_0 = 0, M_4 = 0$.
Составим СЛАУ для нахождения значений $M_i,\ i=1,2,3$:\\\\
$\begin{pmatrix}
2& 0.5& 0\\
0.5& 2& 0.5\\ 
0& 0.5& 2
\end{pmatrix} \cdot
\begin{pmatrix}
M_1\\
M_2\\ 
M_3
\end{pmatrix} = 
\begin{pmatrix}
-2.67633\\
-0.948197\\
2.02381\\
\end{pmatrix}$\\\\
Решая СЛАУ, получим следующее:\\
$M_1 = -1.226,\ M_2 = -0.448,\ M_3 = 1.1240$.\\
Тогда $\tilde{C} = [2.51, 3.95, 2.35, 2.32],\ 
C = [0.25, -0.047,-1.0024, -0.383]$\\\\
Таким образом, получаем функции:
\begin{itemize}
    \item $g_1 = \frac{0}{6 \cdot 1.66}{(6.25-x)}^3 + \frac{-1.226}{6 \cdot 1.66}{(x - 5)}^3 + 0.25 \cdot (x - 5) +2.51 = \\\\ = -1.226 x^3 -0.015x^2 + 0.265x + 5.635$
    \item $g_2 = \frac{-1.226}{6 \cdot 1.66}{(7.5-x)}^3 + \frac{-0.448}{6 \cdot 1.66}{(x - 6.25)}^3 -0.047\cdot (x - 6.25) + 3.95 = \\\\ = 0.081 x^3 +1.802 x^2 - 1.36x - 7.44$
    \item $g_3 = \frac{-0.448}{6 \cdot 1.66}{(8.75-x)}^3 + \frac{1.1240}{6 \cdot 1.66}{(x - 7.5)}^3 -1.0024 \cdot (x - 7.5) + 2.35 = \\\\ = -0.81 x^3 + 6.94 x^2 - 9.88 x -9.39$
    \item $g_4 = \frac{1.1240}{6 \cdot 1.66}{(0-x)}^3 + \frac{0}{6 \cdot 1.66}{(x - 8.75)}^3 - 0.383 \cdot (x - 8.75) + 0.32 = \\\\ = 1.12 x^3 - 1.94 x^2 + 4.13 x +3.67$
\end{itemize}
 Ошибка в неузловой точке:
 \begin{itemize}
     \item $g_1(6) - f(6) = 3.416 - 3.405 = 0.011$
     \item $g_2(7) - f(7) = 5.974 - 3.385 = 2.55$
     \item $g_3(8) - f(8) =  2.785- 2.683 = 0.1$
     \item $g_4(9) - f(9) = 2.359 - 2.089 = 0.27$
 \end{itemize}
\section{Модульная структура программы}
\begin{itemize}
    \item double y(double x) - гладкая функция;
    \item double yn(double x) - негладкая функция;
    \item vector <double> meshGen(double a, double b, int count) - генерация равномерной сетки;
    \item vector <double> hcount(vector <double>& x) - подсчёт коэффициентов $h_i$; 
    \item matrix\_t Mcount(double (*y) (double), vector <double>& x, vector <double>& h, const int& count) - генерация матрицы коэффициентов $M_i$;
    \item vector <double> Thomas(matrix\_t matrix, int count) - метод прогонки;
    \item double Spline(double (*y) (double), const vector <double>& M, const vector <double>& xm, const vector <double>& h, int count, double x) - подсчёт значения сплайна на заданном отрезке в заданной точке;
\end{itemize}
\section{Перечень контрольных тестов}
Требовалось выполнить следующие тесты:
\begin{itemize}
    \item Построить графики самих функций на тестируемом промежутке
    \item Построить сплайн по 6, 10, 16 входных узлах.
    \item Построить график зависимости максимальной ошибки в узлах от их количества
    \item Построить график ошибки в точке на всем отрезке.
    \item Сравнить максимальную ошибку в узлах при интерполяции по формуле Ньютона и сплайнами.
    \item Построить график зависимости максимальной ошибки от разбиения $h = x_{i+1} - x_{i}$.
\end{itemize}
\section{Численный анализ решения задачи}
\textbf{Для гладкой функции:}\\
Из графиков видно, что приближение выполнено корректно и достаточно точно.\\
Чем больше количество входных узлов, тем меньше ошибка при построении. Ошибка убывает гладко.\\
Ошибка в узлах не превосходит ожидаемой.\\

\begin{figure}[h!]
\center
\includegraphics[width=0.6\textwidth]{smooth.png}
\caption{Группа исследований для гладкой функции.}
\end{figure}
\newpage
\newpage
\textbf{Для негладкой функции:}\\
Заметим, что сплайн - гладкая функция по определению. Сплайн хорошо приближает негладкую функцию  точках, не близких к точке разрыва. Приближение в точке разрыва осуществляется неточно.\\
Заметим, что при увеличении количества точек максимальная ошибка убывает, но уже не гладко, как для гладкой функции, а скачкообразно.\\
Ошибка в точках, не близких к точке разрыва, совпадает с гладкой функцией. Вблизи точки разрыва ошибка будет большой, поскольку приближается негладкая функция.
\begin{figure}[h!]
\center
\includegraphics[width=0.6\textwidth]{nsmooth.png}
\caption{Группа исследований для гладкой функции.}
\end{figure}\\
\newpage
\textbf{Сравнение максимальной ошибки Сплайн - Ньютон:}\\
Из графика видно, что сплайн с ростом количества точек сохраняет точность, а метод Ньютона нет. Для негладкой функции ошибка начинает расти с начального количества точек.\\
Ошибка при интерполировании сплайном негладкой функции уменьшается хоть и неравномерно, но уменьшается.
\begin{figure}[h!]
\center
\includegraphics[width=0.4\textwidth]{secondsmooth.png}
\includegraphics[width=0.4\textwidth]{secondnsmooth.png}
\caption{Сравнение ошибки в методах.}
\end{figure}
\newpage
\textbf{Сравнение ошибки в узлах Сплайн-Ньютон:}\\
Заметим, что для достижения одной и той же точности сплайну требуется больше точек, чем полиномам Ньютона.
\begin{figure}[h!]
\center
\includegraphics[width=0.4\textwidth]{newtonsm.png}
\includegraphics[width=0.4\textwidth]{newtonnsm.png}
\includegraphics[width=0.4\textwidth]{ssm.png}
\includegraphics[width=0.4\textwidth]{snsm.png}
\caption{Сравнение ошибки в методах.}
\end{figure}
\newpage
\textbf{Максимальная ошибка от h:}
Из этих графиков видно, что чем мельче разбиение h - тем меньше по модулю ошибка (для сплайна).\\
Для метода Ньютона же заметно, что начиная с некоторого h ошибка растет.\\
Проверим оценку погрешности по формуле $\displaystyle |f(x) - S_3^1(x)| < Ch^4$, где  $\displaystyle c = \frac{5}{384} max|f^{(4)}x|$\\
Построив график $Ch^4$, заметим, что график ошибки для негладкой функции лежит ниже граничного $Ch^4$ и параллелен ему, значит, ошибка метода не превосходит теоретической ошибки и сходится тем же порядком.
\begin{figure}[h!]
\center
\includegraphics[width=0.7\textwidth]{h.png}
\caption{Сравнение ошибки в методах.}
\end{figure}

\section{Краткие выводы}
По проведённым исследованиям я сделал следующие выводы:
\begin{itemize}
    \item Интерполяция методом Ньютона проще в реализации, сплайн более технологичен.
    \item Точность у сплайна и Ньютона примерно одинакова, однако для сплайна для той же точности точек требуется больше.
    \item При небольшом количестве заданных точек лучше использовать метод Ньютона, при большом - сплайн.
    \item Результаты сплайн-интерполяции соответствуют ожидаемым, приближение выполнено корректно, ошибка для гладкой функции уменьшается гладко, для негладкой функции - скачкообразно. 
    \item Вблизи точки разрыва интерполяция сплайном даёт неточный результат, что вполне логично, т.к. сплайн по определению - гладкая функция.
    \item Из этих графиков видно, что чем мельче разбиение h - тем меньше по модулю ошибка (для сплайна).\\
Для метода Ньютона же заметно, что начиная с некоторого h ошибка растет.\\
Результаты соотносятся с графиком зависимости max err от x.\\
Ошибка не превосходит проверенной по формуле.
\end{itemize}
\end{document}
