\documentclass[12pt]{article}
\usepackage[utf8]{inputenc}
\usepackage[T2A]{fontenc}
\usepackage[russian]{babel}
\usepackage[pdftex]{graphicx}
\usepackage{amsmath}
\usepackage{geometry} % Меняем поля страницы
\geometry{left=1.5cm}% левое поле
\geometry{right=2cm}% правое поле
\geometry{top=1cm}% верхнее поле
\geometry{bottom=2cm}% нижнее поле
\graphicspath{{.}}
\DeclareGraphicsExtensions{.pdf,.png,.jpg}

\begin{document}

\begin{titlepage}
\Large

\begin{center}
Санкт-Петербургский \\ Политехнический университет Петра Великого

\vspace{10em}

Отчет по лабораторной работе № 2.1 \\

\vspace{2em}

\textbf{Интерполяция полиномом Ньютона}
\end{center}

\vspace{6em}

\newbox{\lbox}
\savebox{\lbox}{\hbox{Попов Павел Сергеевич}}
\newlength{\maxl}
\setlength{\maxl}{\wd\lbox}
\hfill\parbox{14cm}{
\hspace*{5cm}\hspace*{-5cm}Студент:\hfill\hbox to\maxl{Попов Павел Сергеевич\hfill}\\
\hspace*{5cm}\hspace*{-5cm}Преподаватель:\hfill\hbox to\maxl{Курц Валентина Валерьевна}\\
\\
\hspace*{5cm}\hspace*{-5cm}Группа:\hfill\hbox to\maxl{5030102/00003}\\}

\vspace{\fill}
\begin{center}
Санкт-Петербург \\2022
\end{center}
\end{titlepage}
\section{Формулировка задачи и её формализация}
\textbf{Задача:} Дана гладкая функция $f = \sqrt{x} + cos(x)$, негладкая функция $g = \sqrt{x} + |cos(x)|$. Необходимо выполнить приближение заданных функций при помощи полинома Ньютона, т.е. по данному набору точек $(x_0, x_0), (x_1, x_1) ... (x_n, x_n)$ построить полином $P(x)$, проходящий через заданную систему точек, т.е  $P(x_i) = y_i$, $i = \overline{0 ... n}$. Интерполирование провести в обратном порядке (справа налево), систему точек выбрать свою.

\section{Алгоритм метода и условия применимости}
\subsection{Предварительные вычисления:}
\textbf{Разделенная разность.}\\
Введем понятие разделённой разности.\\
Разделённой разностью первого порядка называется соотношение $\displaystyle y(x_i, x_j) = \frac{y_j - y_i}{x_j - x_i}$.\\
разделённой разностью второго порядка называется соотношение $\displaystyle y(x_i, x_j, x_k) = \frac{y(x_j, x_k) - y(x_i, x_j)}{x_i - x_k} $.\\
Разделённая разность третьего порядка и далее определяются аналогично
рекуррентными соотношениями:\\
$\displaystyle y(x_i, x_j, x_k, x_l) = \frac{y(x_j, x_k, x_l) - y(x_i, x_j, x_k)}{x_l - x_i}$.\\
Вычислим заранее элементы, построим матрицу разделённых разностей.
Вычислить придётся все разделённые разности (n+1 штук, включая значение функции в точках), но для МН интерполяции справа налево потребуются только элементы на диагонали, выделенные красным цветом.\\
\begin{figure}[h!]
\center
\includegraphics[width=0.7\textwidth]{matrix.png}
\caption{Матрица разделённых разностей.}
\end{figure}\\
\textbf{Сетка:}
Для задания полинома по точкам нужно построить сетку. Сетка выбрана свободным образом. Первая и вторая половины точек задаётся формулой $\displaystyle x_i = a + (b-a)\biggr(\frac{i}{n}\biggl)^{p}$, причём при $p > 1$ сгущение происходит при движении к левому концу отрезка. С помощью такого способа сделаем сгущение сетки при приближении к точке разрыва(необходимо для доп. исследования, точка разрыва тестируемой функции $\displaystyle \sqrt{x} + cos(x),  x = \frac{\pi}{2}$. При приближении к правому концу отрезка - $ p < 1$. $a, b$ - границы отрезка для интерполяции.
\subsection{Алгоритм метода:}
Будем искать значение функции в точках, не вычисляя коэффициенты полинома.\\
Формула Ньютона для интерполирования назад: $\displaystyle P_n(x) = y(x_n) + (x-x_n)y(x_n, x_{n-1}) +\\+ (x-x_n)(x-x_{n-1})y(x_n, x_{n-1},x_{n-2}) \cdots (x-x_n)(x-x_{n-1})(x-x_1)y(x_0,x_1,...,x_n) =\\= \sum_{i=0}^{n}y(x_{n-i},x_{n-i+1},..., x_{n})\prod_{k = n-i+1}^{n}(x-x_k)$.


\subsection{Условия применимости:}
 Чтобы полином был единственным степень его должна быть на единицу меньше количества точек ($n+1$ – число точек, $n$ – степень) и все точки должны быть попарно различны.\\
 Непрерывность функции на отрезке (?).
\section{Предварительный анализ задачи}
Функция $\displaystyle \sqrt{x} + cos(x)$ непрерывна по построению на отрезке интерполяции $[1,3]$.
Точки взяты различными. Количество узлов сетки в тестах изменяется от 3 до 50 включительно.
\section{Проверка условий применимости метода}
Все условия применимости автоматически выполнены исходя из построения.
\section{Тестовый пример с расчетами}
$f(x)=\sqrt{x} + cos(x)$, промежуток = $[1,4]$.\\
Равномерная сетка. Разбиение: $x_0=1,x_1=2,x_2=3,x_3=4$.\\
\\
$y_0=1.54$, $y_1=0.998$, $y_2=0.742$, $y_3=1.346,$\\
\\
$y[x_0,x_1]=\frac{y_1-y_0}{x_1-x_0}=-0.542$, $y[x_1,x_2]=\frac{y_2-y_1}{x_2-x_1}=-0.256$, $y[x_2,x_3]=\frac{y_3-y_2}{x_3-x_2}=0,604$\\
\\
$y[x_0,x_1,x_2]=\frac{y[x_1,x_2]y-[x_0,x_1]}{x_2-x_0}=0.143$, $y[x_1,x_2,x_3]=\frac{y[x_2,x_3]-y[x_1,x_2]}{x_3-x_1}=0.430$ \\
\\
$y[x_0,x_1,x_2,x_3]=\frac{y[x_1,x_2,x_3]-y[x_0,x_1,x_2]}{x_3-x_0}=0.095$\\
\\
Тогда $P_n(x)=y_3+y[x_2,x_3](x-x_3)+y[x_1,x_2,x_3](x-x_3)(x-x_2)+y[x_0,x_1,x_2,x_3](x-x_3)(x-x_2)(x-x_1) = 1.346+0.604(x-4)+0.43(x-4)(x-3)+0.095(x-4)(x-3)(x-2)x= 0.095x^3+ 1.291 x^2 + 6.126 x + 0.464$\\
Ошибка в неузловой точке $x=2.4$. |$f(2.4)-p(2.4)|=|0.796-0.818|=0.142$.\\
Достигнута точность $\epsilon = 0.2$\\
Чебышевская сетка: $разбиение: t_k=\cos(\frac{\pi*(2k+1)}{2(n+1)})$, где n-количество узлов $ t_0=0.923, t_1=0.382, t_2=-0.382,t_3=0.923 x_0= 3.973, x_1=3.073, x_2=1.927, x_3= 1.115$.\\
\\
$y_0= 1.319$, $y_1=0.755$, $y_2=y(-1.07)=0.997$, $y_3=1.496$\\
\\
$y[x_0,x_1]=\frac{y_1-y_0}{x_1-x_0}=0.62$, $y[x_1,x_2]=\frac{y_2-y_1}{x_2-x_1}=0.209$, $y[x_2,x_3]=\frac{y_3-y_2}{x_3-x_2}=0.61$\\
\\
$y[x_0,x_1,x_2]=\frac{y[x_1,x_2]-y[x_0,x_1]}{x_2-x_0}=-0.35$, $y[x_1,x_2,x_3]=\frac{y[x_2,x_3]-y[x_1,x_2]}{x_3-x_1}=0.509$ \\
\\
$y[x_0,x_1,x_2,x_3]=\frac{y[x_1,x_2,x_3]-y[x_0,x_1,x_2]}{x_3-x_0}=0.300$\\
\\
Тогда $P_n(x)=y_3+y[x_2,x_3](x-x_3)+y[x_1,x_2,x_3](x-x_3)(x-x_2)+y[x_0,x_1,x_2,x_3](x-x_3)(x-x_2)(x-x_1) = 1.319+0.61(x-1.115)+0.509(x-1.115)(x-1.927) + 0.300(x- 1.115)(x - 1.927)(x - 3.073)=0.3x^3+0.461x^2+3.41x+4.213$\\
Ошибка в неузловой точке $x=2.4$. |$f(2.4)-p(2.4)|=|0.8136-0.8153|=0.01$.\\
Из этого сравнения можно сделать вывод, что чебышевская сетка реализует лучшее приближение, чем равномерная.
\section{Модульная структура программы}
\begin{itemize}
    \item double function(double x); Реализует непосредственно функцию $\sqrt{x} + cos(x)$.
    \item double nonSmooth(double x); Реализует негладкую функцию $\sqrt{x} + |cos(x)|$.
    \item vector <double> divDifference(double(*function) (double), vector <double> mesh). Создаёт матрицу разделённых разностей и возвращает нужный массив с элементаами на диагонали.
    \item vector <double> divDifference(double(*function) (double), vector <double> mesh). Генерирует сетку для построения полинома.
    \item vector <double> linspaceGen(int a, int b, int number). Генерирует равномерную сетку для построения полинома по значениям функции в точке.
    \item double NewtonRightLeft(vector <double> divMas, vector <double> mesh, double x). Реализует непосредственно метод Ньютона.
\end{itemize}
\section{Перечень контрольных тестов}
Необходимо выполнить следующие исследования для гладкой и негладкой функции.\\
\begin{itemize}
    \item Построить графики тестируемых функций;
    \item Построить полученные полиномы 3, 5, 7 степеней;
    \item Построить график зависимости максимальной ошибки от количества узлов;
    \item Построить график зависимости ошибки в linspace-точках графика.
\end{itemize}
\section{Численный анализ решения задачи}
\begin{figure}[h!]
\center
\includegraphics[width=0.7\textwidth]{smooth.png}
\caption{Группа исследований для гладкой функции.}
\end{figure}\\
Из графиков видно, что для гладкой функции:

\begin{itemize}
    \item Получены полиномы 3, 5, 7 степеней. Полиномы приближённо совпадают с графиком исходной функции.\\
    \item Максимальная ошибка до определенного момента уменшается и достигает порядка $10^{-7}$, с некоторого момента ($~20 $ узлов в сетке) ошибка начинает возрастать. Результат соответствует ожидаемому. Не самая лучшая точность объясняется выбором не самой лучшей сетки. Точность на равномерной сетке достигает порядка $10^{-12}$.
    \item По графику ошибки видно, что в узловых точках ошибка имеет наименьшее значение. Чем выше степень полинома, тем ниже график ошибки, тем меньше ошибка. Результат ожидаемый.
\end{itemize}
\begin{figure}[h!]
\center
\includegraphics[width=0.7\textwidth]{nsmooth.png}
\caption{Группа исследований для негладкой функции.}
\end{figure}
Для негладкой функции сгущение количества точек осуществлялось в точку разрыва. При этом видно, что:\\
\begin{itemize}
    \item Построены графики исходной функции и приближённых полиномов. По ним видно, что полиномы совпадают с тестируемой функцией примерно наполовину. Приближение некачественное.
    \item Начиная с минимального количества узлов максимальная ошибка сразу возрастает. Это можно объяснить разрывом производной в точке $\frac{\pi}{2}$.
    \item Результат графика ошибки во всех выбранных точках в 10 раз хуже, чем для гладкой функции.
\end{itemize}
\section{Краткие выводы}
В данной работе я провёл исследование приближения гладкой и негладкой функции.\\
Получено, что для гладкой функции на выбранном отрезке приближение соответствует ожидаемому, однако для негладкой функции на отрезке приближения разрывов производной быть не должно. Это существенно ухудшает качество приближения.\\
Также стоит отметить, что качество приближения зависит от выбранной сетки. Выбранная мною сетка объективно даёт не лучший результат.
\end{document}