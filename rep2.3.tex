\documentclass[12pt]{article}
\usepackage[utf8]{inputenc}
\usepackage[T2A]{fontenc}
\usepackage[russian]{babel}
\usepackage[pdftex]{graphicx}
\usepackage{amsmath}
\usepackage{geometry} % Меняем поля страницы
\geometry{left=1.5cm}% левое поле
\geometry{right=2cm}% правое поле
\geometry{top=1cm}% верхнее поле
\geometry{bottom=2cm}% нижнее поле
\graphicspath{{.}}
\DeclareGraphicsExtensions{.pdf,.png,.jpg}

\begin{document}

\begin{titlepage}
\Large

\begin{center}
Санкт-Петербургский \\ Политехнический университет Петра Великого

\vspace{10em}

Отчет по лабораторной работе №2.3\\

\vspace{2em}

\textbf{Численное интегрирование с помощью формул
Ньютона-Котеса. Формула средних прямоугольников.}
\end{center}

\vspace{6em}

\newbox{\lbox}
\savebox{\lbox}{\hbox{Попов Павел Сергеевич}}
\newlength{\maxl}
\setlength{\maxl}{\wd\lbox}
\hfill\parbox{14cm}{
\hspace*{5cm}\hspace*{-5cm}Студент:\hfill\hbox to\maxl{Попов Павел Сергеевич\hfill}\\
\hspace*{5cm}\hspace*{-5cm}Преподаватель:\hfill\hbox to\maxl{Курц Валентина Валерьевна}\\
\\
\hspace*{5cm}\hspace*{-5cm}Группа:\hfill\hbox to\maxl{5030102/00003}\\}

\vspace{\fill}
\begin{center}
Санкт-Петербург \\2022
\end{center}
\end{titlepage}
\section{Формулировка задачи и её формализация}
\textbf{Задача:} Вычислить определенный интеграл $I$ функции $f(x) = \sqrt{x} + cos(x)$ на промежутке $[a, b] = [0.1,1]$ с помощью квадратурной формулы Ньютона-Котеса (средние прямоугольники).
\section{Алгоритм метода и условия его применимости} 
\textbf{Алгоритм метода:} \\ 
Из мат. анализа известно, что для функций, допускающих на промежутке $[a,b]$ конечное число точек разрыва первого рода можно вычислить значение интеграла по определению:\\
\begin{center}
    $\displaystyle I=\lim _{n \rightarrow \infty} \sum_{i=1}^{n} f\left(\xi_{i}\right)\left(x_{i}-x_{i-1}\right)$
\end{center}
где $\left\{x_{i}\right\}_{i=0}^{n}-$ произвольная упорядоченная система точек отрезка $[a, b]$ такая, что
$$
 \displaystyle \max \left\{x_{0}-a, x_{i}-x_{i-1}, b-x_{n}\right\} \rightarrow 0 \quad \text { при } n\rightarrow \infty
$$
$
\text { а } \xi_{i} \text { - произвольная точка элементарного промежутка }\left[x_{j-1}, x_{i}\right] \text {. }
$
Зафиксировав некоторое $n \geq 1$, будем иметь 
$$
I \approx \sum_{i=1}^{n} f\left(\xi_{i}\right)\left(x_{i}-x_{i-1}\right) .
$$
Это приближенное равенство назовем общей формулой прямоугольников (площадь криволинейной трапеции приближенно заменяется плющадью ступенчатой фигуры, составленной из прямоугольниковб основаниями которых служат отрезки $x[_{i-1},x_i]$, а высотами - ординаты $f(\xi_i)$.\\
Разбиение отрезка $\displaystyle [a,b]$ выберем равномерное: $\displaystyle h = \frac{b-a}{n}$, полагая
$$
x_0 = a, x_i = x_{i-1} + h, (i = 1,2, ..., n-1), x_n = b
$$
При таком разбиении формула примет вид
$$
I \approx h \sum_{i=1}^{n} f\left(\xi_{i}\right), \quad \xi_{i} \in\left[x_{i-1}, x_{i}\right]
$$
Положим теперь 
$$
\xi_i = \frac{1}{2}(x_{i-1}+x_i) = x_{i-1}+\frac{h}{2} = x_i - \frac{h}{2}
$$
В результате получим формулу средних прямоугольников, остаточный член которой равен:
$$
r^{\Pi}(h)=\frac{b-a}{24} f^{\prime \prime}\left(\xi_{n}\right) h^{2}, \quad \xi_{I} \in(a, b) .
$$
\textbf{Условия применимости:}\\
$f \in C^2$.\\
$f$ ограничена на $[a,b]$.
\section{Предварительный анализ задачи}
Функция $f(x) = \sqrt(x) + cos(x)$ дифференцируема, очевидно, требуемое количество раз.\\
Непрерывна по построению на заданном отрезке.
\section{Проверка условий применимости метода}
Условия применимости выполнены автоматически исходя из выбора функции.
\section{Тестовый пример с расчетами}
$f(x) = \sqrt{x} + cos(x)$.\\
Имеем одну точку (один промежуток): $[0.1, 1]$, $h = 0.9$\\
$I_1 = 0.9 * f(0.1 + 0.45 * 0 + \frac{0.45}{2}) = 1.43472993$.\\\\
Имеем 2 точки (2 промежутка): $[0.1, 0.55, 1], h = 0.45$.\\
$\displaystyle I_2 = 0.45\cdot f(0.1 + 0.45 \cdot 0 + \frac{0.45}{2}) + 0.45\cdot f(0.1 + 0.45 \cdot 0 + \frac{0.45}{2}) = 1.400625$.\\\\
Применим правило Рунге:
$$
err_1 = \frac{|I_2-I_1|}{2^2-1} =0.0341 
$$
Имеем 4 точек (4 промежутка): $[0.1, 0.325, 0.55, 0.775, 1], h = 0.225$.\\ 
$\displaystyle I_4 = 0.225\cdot f(0.1 + 0.225 \cdot  0 + \frac{0.225}{2}) + 0.45\cdot f(0.1 + 0.45 \cdot 1 + \frac{0.45}{2}) + 0.45\cdot f(0.1 + 0.45 \cdot 2 + \frac{0.45}{2}) + 0.225·f (0.1+0.225\cdot3+ \frac{0.225}{2}) = 1.390850.\\
$
Применим правило Рунге:\\
$$
err_2 = \frac{|I_4-I_2|}{2^2-1} =0.009775
$$\\
Тогда 
$$
\frac{err_2}{err_1} = \frac{0.032581}{0.009775} = 3,58 \sim 4.
$$
Отношение показывает, что при увеличении количества узлов в 2 раза ошибка уменьшилась в $\sim 2^2$ раз, что соответствует порядку сходимости метода $n = 2$.
\section{Модульная структура программы}
\begin{itemize}
    \item double y(double x) Реализует саму функцию.
    \item double Integrate(double a, double b, int count, double (*f) (double)); Реализует интегрирование;
    \item void AbsErrPlot() Выводит данные для графиков;
    \item void NodesEpsPlot() Выводит данные для графиков;
\end{itemize}
\section{Перечень контрольных тестов}
Требуется построить следующие графики:
\begin{itemize}
    \item График зависимости абсолютной погрешности от количества интервалов
    \item График зависимости количества интервалов для требуемого эпсилон
    \item Графики зависимостей теоретической и практической ошибки от требуемой точности.\\
\end{itemize}
\section{Численный анализ задачи}

\begin{figure}[h!]
\center
\includegraphics[width=0.6\textwidth]{err_eps.png}
\includegraphics[width=0.6\textwidth]{err_eps_enlarged.png}
\caption{Фактическая, теоретическая ошибки от заданного $\varepsilon$}
Из граика видно, что фактическая ошибка (разность точного и полученного значения) лежит в теоретических пределах, полученная ошибка не превосходит зананной точности.\\
\end{figure}

\begin{figure}[h!]
\center
\includegraphics[width=0.6\textwidth]{N_e.png}
\caption{Количество интервалов для требуемой точности по правилу Рунге.}
Заметим, что для большей заданной точности требуется большее количество узлов.Количество узлов увеличивалось в соответствии с правилом Рунге.\\
\end{figure}

\begin{figure}[h!]
\center
\includegraphics[width=0.6\textwidth]{runge_err_eps.png}
\caption{количество интервалов от абсолютной погрешности}
Из графика видно, что полученная по правилу Рунге ошибка не превосходит заданной. Метод работает верно.\\
\end{figure}
\newpage
\section{Краткие выводы}
В данной работе я провёл исследвоание вычисления интеграла с помощью формулы средних прямоугольников.\\
Было установлено, что:
\begin{itemize}
    \item За каждое увеличение число узлов в $i$ раз по правилу Рунге ошибка уменьшается в $2^i$ раз.
    \item С увеличением числа узлов точность увеличивается.
    \item фактическая ошибка не нарушает установленных теоретических пределов.
\end{itemize}
\end{document}
