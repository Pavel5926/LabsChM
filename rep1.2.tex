\documentclass[12pt]{article}
\usepackage[utf8]{inputenc}
\usepackage[T2A]{fontenc}
\usepackage[russian]{babel}
\usepackage[pdftex]{graphicx}
\usepackage{amsmath}
\usepackage{geometry} % Меняем поля страницы
\geometry{left=1.5cm}% левое поле
\geometry{right=2cm}% правое поле
\geometry{top=1cm}% верхнее поле
\geometry{bottom=2cm}% нижнее поле
\graphicspath{{.}}
\DeclareGraphicsExtensions{.pdf,.png,.jpg}

\begin{document}

\begin{titlepage}
\Large

\begin{center}
Санкт-Петербургский \\ Политехнический университет Петра Великого

\vspace{10em}

Отчет по лабораторной работе №2\\

\vspace{2em}

\textbf{Решение СЛАУ прямыми методами}
\end{center}

\vspace{6em}

\newbox{\lbox}
\savebox{\lbox}{\hbox{Корнелюк Алексей Викторович}}
\newlength{\maxl}
\setlength{\maxl}{\wd\lbox}
\hfill\parbox{14cm}{
\hspace*{5cm}\hspace*{-5cm}Студент:\hfill\hbox to\maxl{Попов Павел Сергеевич\hfill}\\
\hspace*{5cm}\hspace*{-5cm}Преподаватель:\hfill\hbox to\maxl{Курц Валентина Валерьевна}\\
\\
\hspace*{5cm}\hspace*{-5cm}Группа:\hfill\hbox to\maxl{5030102/00003}\\}

\vspace{\fill}
\begin{center}
Санкт-Петербург \\2021
\end{center}
\end{titlepage}

\section{Формулировка задачи и её формализация}
\textbf{Задача:}
Найти решение $X$ системы линейных алгебраических уравнений $AX = B$.\\
Решение тестовой СЛАУ с 10 неизвестными найдём с помощью "Метода вращений". 

\section{Алгоритм метода и условия применимости}

\subsection{Алгоритм:}\\ 
Алгоритм делится на две подзадачи: прямой ход Методом вращений, с помощью которого приведем матрицу к верхнетреугольному виду и обратный ход Методом Гаусса, с помощью которого последовательно выразим переменные.
\textbf{Прямой ход методом вращений:}\\
Пусть имеется СЛАУ, заданная расширенной матрицей:
$\begin{pmatrix}
a_{11}& a_{12}& \cdots& a_{1n}& $\vrule$ &b_{1}\\
a_{21}& a_{22}& \cdots& a_{2n}& $\vrule$ &b_{2}\\
\vdots& \vdots& \ddots& \vdots& $\vrule$ &\vdots\\
a_{n1}& a_{n2}& \cdots& a_{nn}& $\vrule$ &b_{n}\\
\end{pmatrix}$
Пусть $c_{12}$ и $s_{12}$ - некоторые отличные от нуля числа. Заменим первое уравнение линейной комбинацией 1-го и 2-го уравнений, умноженных на $c_{12}$ и $s_{12}$ соответственно.
Новое 2-е уравнение - линейная комбинация 1-го и 2-го уравнений с коэффициентами $-s_{12}$ и $c_{12}$.
\begin {center} $(c_{12}a_{11}+s_{12}a_{21})x_{1} + \cdots + (c_{12}a_{1n}+s_{12}a_{2n})x_{n} = c_{12}b_{1}+s_{12}b_{2}$\\
$(-s_{12}a_{11}+c_{12}a_{21})x_{1} + \cdots + (-s_{12}a_{1n}+c_{12}a_{2n})x_{n} = -s_{12}b_{1}+c_{12}b_{2}$\\
\end {center}
Найдем такие $c_{12}$ и $s_{12}$, чтобы они удовлетворяли условиям:
\begin{center}
$-s_{12}a_{11}+c_{12}a_{21} = 0$ и $s_{12}^2 + c_{12}^2 = 1$\\ 
\end{center}
Тогда:
\begin{center}
$c_{12} = \frac{a_{11}}{\sqrt{a_{11}^2+a_{21}^2}}$, $s_{12} = \frac{a_{21}}{\sqrt{a_{11}^2+a_{21}^2}}$
\end{center}
Таким образом система примет вид:
\begin{center}
$\begin{pmatrix}
a_{11}^{(1)}& a_{12}^{(1)}& \cdots& a_{1n}^{(1)}& $\vrule$ &b_{1}^{(1)}\\
0& a_{22}^{(1)}& \cdots& a_{2n}^{(1)}& $\vrule$ &b_{2}^{(1)}\\
\vdots& \vdots& \ddots& \vdots& $\vrule$ &\vdots\\
a_{n1}& a_{n2}& \cdots& a_{nn}^{(1)}& $\vrule$ &b_{n}^{(1)}\\
\end{pmatrix}$\\
\end{center}
С помощью данных преобразований мы фактически "обнулили"\ коэффициент $a_{21}$ СЛАУ.\\
Проделаем $n-1$ аналогичных преобразований, обнуляя i-ю переменную первого столбца тогда придём к системе:\\
\begin{center}
$\begin{pmatrix}
a_{11}^{(n-1)}& a_{12}^{(n-1)}& \cdots& a_{1n}^{(n-1)}&$\vrule$&b_{1}^{(n-1)}\\
0& a_{22}^{(n-1)}& \cdots& a_{2n}^{(n-1)}& $\vrule$ &b_{2}^{(n-1)}\\
\vdots& \vdots& \ddots& \vdots& $\vrule$ &\vdots\\
0& a_{n2}^{(n-1)}& \cdots& a_{nn}^{(n-1)}& $\vrule$ &b_{n}^{(n-1)}\\
\end{pmatrix}$\\
\end{center}
Аналогичным образом исключим переменную под номером 2 из всех уравнений СЛАУ за $n-2$ шагов
\begin{center}
$\begin{pmatrix}
a_{11}^{(n-2)}& a_{12}^{(n-2)}& \cdots& a_{1n}^{(n-2)}& $\vrule$ &b_{1}^{(n-2)}\\
0& a_{22}^{(n-2)}& \cdots& a_{2n}^{(n-2)}& $\vrule$ &b_{2}^{(n-2)}\\
\vdots& \vdots& \ddots& \vdots& $\vrule$ &\vdots\\
0& 0& \cdots& a_{nn}^{(n-2)}& $\vrule$ &b_{n}^{(n-2)}\\
\end{pmatrix}$\\
\end{center}
В результате $n-1$ таких этапов прямого хода придём к верхнетреугольной матрице:
\begin{center}
$\begin{pmatrix}
a_{11}^{(n-1)}& a_{12}^{(n-1)}& \cdots& a_{1n}^{(n-1)}& $\vrule$ &b_{1}^{(n-1)}\\
0& a_{22}^{(n-1)}& \cdots& a_{2n}^{(n-1)}& $\vrule$ &b_{2}^{(n-1)}\\
\vdots& \vdots& \ddots& \vdots& $\vrule$ &\vdots\\
0& 0& \cdots& a_{nn}^{(n-1)}& $\vrule$ &b_{n}^{(n-1)}\\
\end{pmatrix}$\\
\end{center}
В общем случае преобразования можно описать так в матричном виде:

\begin{center}
    $A^{(1)}X = B^{(1)}$\\
    $A^{(1)} = T_{1n}\cdots T_{13} T_{12} B^{(1)}$
\end{center}
где 
\begin{center}
$T_{ij}(\phi) =$
$\begin{pmatrix}
c_{ij}& s_{ij}& \cdots& 0& \\
-s_{ij}& c_{ij}& \cdots& 0&\\
\vdots& \vdots& \ddots& \vdots& \\
0& 0& \cdots& 1& \\
\end{pmatrix}$
\end{center}
матрица поворота, задаваемая номерами исключаемого переменного i из уравнения j  и углом поворота $\phi$.\\
Стоит отметить, что норма любого вектор-столбца расширенной матрицы системы остается такой же, как у соответствующего столбца исходной системы.\\
\textbf{Обратный ход методом Гаусса:}
Выразим столбец, являющийся решением СЛАУ, при помощи метода Гаусса.\\
$x_{k} = \frac{1}{a_{kk}^{(n-1)}}\bigg(b_{k}^{(n-1)}- \sum\limits_{j=k+1}^na_{kj}^{(n-1)}x_{j}\bigg)$\\ 
где $x_{k}$ - соответственно $k$-я компонента искомого вектора $X$. 
\subsection{Условия применимости:}
Требование: матрица коэффициентов СЛАУ не является вырожденной, т.е. $\det(A) \neq 0$.\\
Это следует из условий построения матрицы (A = $Q \cdot D \cdot Q'$).\\
$Q$ - ортогональная, а значит невырожденная матрица.\\
$D$ - диагональная, с ненулевыми элементами на диагонали - тоже невырожденная. \\
\\Значит, их произведение - невырожденная матрица $A$.\\
Тогда система имеет решение, причём единственное.
\section{Предварительный анализ задачи}
Матрица коэффициентов СЛАУ не является вырожденной, т.е. $\det(A) \neq 0$.\\
Условие выполняется автоматически при выбранном способе  генерации тестируемых матриц А.

\section{Проверка условий применимости метода}
Условие выполняется автоматически при выбранном способе  генерации тестируемых матриц А.
\section{Тестовый пример с расчетами}
Для удобства вычислений выберем $\epsilon = 0.1$

\begin{equation*}
 \begin{cases}
   x + y + 2z = 5 
   \\
   -x -y +2z = -1
   \\
   x + z = 2
 \end{cases}
\end{equation*}

Запишем в матричном виде, проведем прямой ход:\\
$\begin{pmatrix}
1& 1& 2& $\vrule$ &5\\
-1& -1& 2& $\vrule$ &-1\\
1& 0& 1& $\vrule$ &2\\
\end{pmatrix}\rightarrow$
$\begin{pmatrix}
\sqrt{2}& \sqrt{2}& 0& $\vrule$ &3\sqrt{2}\\
0& 0& 2\sqrt{2}& $\vrule$ &2\sqrt{2}\\
1& 0& 1& $\vrule$ &2\\
\end{pmatrix}\rightarrow$
$\begin{pmatrix}
\sqrt{3}& \frac{2}{\sqrt{3}}& \frac{1}{\sqrt{3}}& $\vrule$ &\frac{8}{\sqrt{3}}\\
0& -\sqrt{\frac{2}{3}}& +\sqrt{\frac{2}{3}}& $\vrule$ &-\sqrt{\frac{2}{3}}\\
0& 0& -2\sqrt{2}& $\vrule$ &-2\sqrt{2}\\
\end{pmatrix}$\\
Значения $cos$ и $sin$ на шагах:\\
\begin{enumerate}
    \item  $cos\alpha_1 = \frac{\sqrt{2}}{2}$ $sin\alpha_1 = -\frac{\sqrt{2}}{2}$;
    \item  $cos\alpha_2 = \sqrt{\frac{2}{3}}$ $sin\alpha_2 = \frac{\sqrt{3}}{3}$;
    \item  Значение эл-та матрицы $a_{31}$ уже нулевое, можно переходить к обратному ходу.
\end{enumerate}
Теперь проведем обратный ход методом Гаусса:
\begin{enumerate}
\item $z = 1$
\item $y = 2$
\item $x+y+2z=5 \Rightarrow x = 1$
\end{enumerate}
Получили столбец-решение $X = \begin{pmatrix}
1\\
2\\
1\\
\end{pmatrix}$

\section{Перечень контрольных тестов}
В качестве проверки полученных результатов построим три зависимости:\\
\begin{enumerate}
    \item График зависимости относительной погрешности $\frac{|| x - x^{*} ||}{|| x^{*} ||}$ вычислений от выбранного числа обусловленности $cond \in [10,10^{10}]$ с шагом в 10, где $x^{*}$ - точное решение, сгенерированное в MATLAB, а $x$ - решение, полученное реализованным методом на Си.\\
    \item График зависимости относительной погрешности $\frac{|| x - x^{*} ||}{|| x^{*} ||}$ вычислений от самостоятельно внесённой ошибки в вектор-столбец $\Tilde{B}$, где $\Tilde{B} = B +\delta B$, $B = AX$.\\
    \item График зависимости скорости работы метода от размерности тестовых матриц. Измеряется при помощи подключения библиотеки <windows.h>.
\end{enumerate}


\newpage
\section{Модульная структура программы}
\begin{itemize}
\item Константа, определенная через \#define, нужная для теста измерения времени.\\
\#define  NUM\_DIMENSION 2000\\

\item Функция double* VectorInit(int dimension);\\
Создает(инициализирует) нуль-вектор.\\
Принимает на вход размерность, возвращает указатель на массив double*.\\
\item Функция void VectorDelete(double* vector);\\
Освобождает память, ранее занятую под вектор.\\
Принимает на вход указатель на вектор, ничего не возвращает.\\
\item  Функция void PrintVectorScreen(double* vector, int dimension);\\
Печатает вектор на экран. Отладочная функция.\\
Принимает на вход указатель на вектор и его размерность. Ничего не возвращает.\\
\item  Функция void PrintMatrixScreen(double** vector, int dimension);\\
Печатает матрицу на экран. Отладочная функция.\\
Принимает на вход указатель на матрицу и его размерность. Ничего не возвращает.\\
\item Функция double** ParseMatrix(FILE* f, double** matrix, int dimension);\\
Считывает матрицу из файла.\\
Принимает на вход поток файла, указатель на матрицу, возвращает указатель на матрицу double**.\\
\item Функция double** ParseVector(FILE* g, double* vector, int dimension);\\
Считывает вектор из файла.\\
Принимает на вход поток файла, указатель на вектор, возвращает указатель на вектор double*.\\
\item  Функция void PrintVector(FILE* g, double* vector, int dimension);\\
Печатает вектор в файл.\\
Принимает на вход поток файла, указатель на вектор и его размерность. Ничего не возвращает.\\
\item Функция double** MergeMatrix(double** A, double* B, int dimension);\\
Обеспечивает слияние матрицы $A$ и вектор-столбца $B$, возвращает расширенную матрицу СЛАУ.\\
Принимает на вход матрицу $A$ и вектор-столбец $B$.\\
\item Функция double* RotationsMethod(double** A, double* B, int dimension);\\
Реализует непосредственно метод вращений.\\
Принимает на вход матрицу $A$, вектор-столбец $B$ и размерность решения $X$.Возвращает полученное методом решение.
\end{itemize}











\section{Численный анализ решения задачи}
\subsection{Зависимость относительной погрешности от числа обусловленности матрицы $A$}
\begin{figure}[h!]
\center
\includegraphics[width=1\textwidth]{rel_err_cond.png}
\caption{Зависимость погрешности от числа обусловленности}
\label{error_cond}
\end{figure}
Из графика ясно видно, что с ростом числа обусловленности растет и ошибка. Это ожидаемый результат для данного численного метода.

\newpage
\subsection{Зависимость относительной погрешности от погрешности входных данных}
\begin{figure}[h!]
\center
\includegraphics[width=1\textwidth]{rel_err_db.png}
\caption{Зависимость погрешности от ошибки во входных данных}
\label{error_errorData}
\end{figure}
График показывает, что с увеличением ошибки во входных данных линейно возрастает относительная погрешность. Это ожидаемый результат, поскольку метод обладает хорошей обусловленностью.
\newpage
\subsection{Зависимость времени обработки данных от размерности входной матрицы}
\begin{figure}[h!]
\center
\includegraphics[width=1\textwidth]{time.png}
\caption{Зависимость времени от размерности}
\label{error_errorData}
\end{figure}
График показывает, что время работы соответствует ожидаемому $O(\alpha n^3)$. Приближенная кривая получена с помощью cftool MATLAB и ее уравнение:\\ $f(x) = 1.472e^{-9}x^3+1.616e^{-6}x^{2} -0.001022x +0.1352$. \\Результат ожидаемый, т.к. метод вращений работает медленнее, чем метод Гаусса.
\newpage
\section{Краткие выводы}
В данной работе я проанализировал решение СЛАУ методом вращений.\\
В исследование входило: реализация самого метода на языке СИ, генерация входных данных для Си в MATLAB, построение графиков в MATLAB.\\
Исследованием подтверждено, что с увеличением числа обусловленности входной матрицы растет и погрешность в вычислениях.\\
С линейным ростом ошибки во входных данных наблюдается линейный рост в относительной погрешности.\\ 
Время работы метода составляет $O(\alpha n^3)$.\\
Из положительных сторон метода следует отметить, что при умножении матрицы поворота на вектор-столбец СЛАУ норма вектор-столбца остается равна исходной.\\
Метод обладает хорошей обусловленностью.\\
Из отрицательных сторон стоит отметить то, что время работы довольно долгое (даже дольше, чем метод Гаусса).\\
\end{document}