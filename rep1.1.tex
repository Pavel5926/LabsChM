\documentclass[12pt]{article}
\usepackage[utf8]{inputenc}
\usepackage[russian]{babel}
\usepackage[pdftex]{graphicx}
\usepackage{ wasysym }
\graphicspath{{.}}
\usepackage{xcolor}
\usepackage{hyperref}
\DeclareGraphicsExtensions{.pdf,.png,.jpg}
\usepackage{geometry} % Меняем поля страницы
\geometry{left=2cm}% левое поле
\geometry{right=3cm}% правое поле
\geometry{top=1cm}% верхнее поле
\geometry{bottom=1.5cm}% нижнее поле

\begin{document}
\begin{titlepage}
\Large

\begin{center}
Санкт-Петербургский \\ Политехнический университет Петра Великого

\vspace{10em}

Отчет по лабораторной работе №1\\

\vspace{2em}

\textbf{Решение алгебраических и трансцендентных уравнений}
\end{center}

\vspace{6em}

\newbox{\lbox}
\savebox{\lbox}{\hbox{Попов Павел Сергеевич}}
\newlength{\maxl}
\setlength{\maxl}{\wd\lbox}
\hfill\parbox{12cm}{
\hspace*{3cm}\hspace*{-5cm}Студент:\hfill\hbox to\maxl{Попов Павел Сергеевич\hfill}\\
\hspace*{3cm}\hspace*{-5cm}Преподаватель:\hfill\hbox to\maxl{Курц Валентина Валерьевна}\\
\\
\hspace*{3cm}\hspace*{-5cm}Группа:\hfill\hbox to\maxl{5030102/00003}\\
}

\vspace{\fill}

\begin{center}
Санкт-Петербург \\2021
\end{center}

\end{titlepage}

\section{Формулировка задания и его формализация}
\textbf{Задача:}\\Решение алгебраического и трансцендентного уравнений на заданных промежутках с определенной точностью. Поиск корней осуществляется при помощи "Метода половинного деления" и "Метода секущих".\\
Алгебраическое уравнение: $4x^5-8x^4-x^2+3x+5 = 0$ Отрезок: $[1.7,2.0]$ (МПД), [2.0,2.25] (МС)\\
Трансцендентное уравнение $x*cos(x)+3x^2-1=0$ Отрезок: $[0.25,0.5]$ (МПД), [0.5,0.75] (МС)

\section{Алгоритмы методов и условия его применимости}
\subsection{Метод половинного деления}
\textbf{Алгоритм}: Пусть $c = \frac{(a+b)}{2}$. Вычислим $f(c)$. Если $f(c)=0$, то $c$ - корень. Если $f(c)\neq0$, проверяем условие $f(a)*f(c)$. Если $f(a)*f(c)<0$, то $b=c$, иначе $a=c$. Итерационный процесс продолжается до тех пор, пока $|b-a|<2\epsilon$.  \\
\textbf{Условие применимости:}\\
1. $f(x)\in C([a,b])$ т.е. $f(x)$ непрерывна на $[a,b]$.\\
2. $f(a)*f(b)<0$.\\
\subsection{Метод секущих}\\
\textbf{Алгоритм:} Выберем два начальных приближения $prev$ и $cur$, заготовим переменную под приближение $next$. Приближения должны быть по одну сторону от корня, $cur < prev$, т.к. подходим к корню справа. Далее начнем сам процесс приближения:\\
$next = prev$;\\
$prev = cur$;\\
$cur = cur + (cur - next) / (f(next) / f(cur) - 1)$;\\
Будем выполнять его до момента пока истинно условие выполнения $(M2 / (2 * m1) * |next - cur| * |cur - prev| > \epsilon)$, где $M2$ и $m1$ - $max f"(x)$ и $min f'(x)$  на $[a,b]$ соответстввенно.\\
Дополнительно приведу формулу для метода секущих: $x^{k+1}=\frac{x^{k}-x^{k-1}}{f(x^{k})-f(x^{k-1})}$.
\textbf{Условие применимости:}\\
1. $f(x)\in C([a,b])$ т.е. $f(x)$ непрерывна на $[a,b]$.\\
2.$f'(x)$,$f"(x)$ знакопостоянны на $[a,b]$.\\
3.$f'(x)$ непрерывна и отлична от нуля на $[a,b]$.\\
4. Первые два приближения удовлетворяют условию Фурье: $f(x^{(i)})*f"(x^{(i)})>0,$ i = {0,1}.\\
\section{Предварительный анализ задачи}
Нахождение верхних и нижних границ алгебраического уравнения.\\
\textbf{Для положительного корня:}\\
\[x^* \leq 1+\sqrt[1]{\frac{8}{4}} = 3\]
Сделав замену $x=\frac{1}{y}$, получим:\[\frac{4}{y^5}-\frac{8}{y^4}-\frac{1}{y^2}+\frac{3}{y}+5 = 0 \]
\[ 5y^5+3y^4-y^3-8y^4+4=0\]
\[y\leq 1+\sqrt{\frac{8}{5}} \approx 2.261\]\\
\[x^{*} = \frac{1}{y} \approx 0.44 \]\\
Для положительных корней имеем промежуток $x^{*} \in[0.44,3]$.\\
\textbf{Для отрицательного корня:}\\
Сделаем замену $x =-y$:\\
\[ y \le 1+\sqrt[1]{8/4}\]
\[ x^* \ge -3 \]
Сделаем замену $x = \frac{-1}{y}$:
\[ y \le 1+\sqrt[1]{8/4}\]
\[x^* \le \frac{-1}{3}\]\
Для отрицательных корней имеем промежуток: $x^*\in[-3,\frac{-1}{3}$]
\section{Проверка условий применимости метода}
\subsection{Метод половинного деления}
\begin{figure}[h!]
\center{\includegraphics[width=0.6\textwidth]{Графики.png}}
\caption{Алгебраическая и трансцендентная функции}
\end{figure}\\
Из рисунка №1 видно, что функции имеют только один корень на рассматриваемых отрезках.\\
Из рисунка №1 видно, что корень алгебраического уравнения находится на промежутке $[1.7,2]$, а для трансцендентной - $[0.25,0.5]$.\\
\textbf{Для алгебраического уравнения:}\\
Из рисунка №1 видно, что на данном отрезке функция монотонна и на концах отрезка она принимает значения разных знаков.\\
\textbf{Для трансцендентного уравнения:}\\
Из рисунка №1 видно, что на данном отрезке функция монотонна и на концах отрезка она принимает значения разных знаков.\\
\subsection{Метод секущих}
\textbf{Для трансцендентного уравнения:}\\ 
Мы выбрали отрезок $[0.5,0.75]$.\\
Из рисунка №1 видно, что на данном отрезке функция монотонна.\\
\begin{figure}[h!]
\center{\includegraphics[width=0.6\textwidth]{тр_первая.png}}
\caption{Трансцендентная функция. $f'(x)$.}
\end{figure}\\
\begin{figure}[h!]
\center{\includegraphics[width=0.6\textwidth]{тр_вторая.png}}
\caption{Трансцендентная функция. $f''(x)$.}
\end{figure}\\
1. Из рисунков 2,3 видно, что на выбранном промежутке $f'(x)$,$f"(x)$ производные существуют и знакопостоянны.\\
2. $f'(x)$ существует и отлична от нуля, что видно из графика №2.\\
3. Проверим условие Фурье для точек x = 0.5, x = 0.75:\\
$(6 - x cos(x) - 2 sin(x))*(x*cos(x)+3x-1)\approx 4.32 > 0, x = 0.5$\\
$(6 - x cos(x) - 2 sin(x))*(x*cos(x)+3x-1)\approx 7.35 > 0, x = 0.75$\\
Оба начальных приближения соответствуют условиями применимости, метод можно применить.\\
\textbf{Для алгебраического уравнения:}\\
\begin{figure}[h!]
\center{\includegraphics[width=0.6\textwidth]{Полином.png}}
\caption{Алгебраическая функция. $f(x)$.}
\end{figure}\\
\begin{figure}[h!]
\center{\includegraphics[width=0.6\textwidth]{Поли_первая.png}}
\caption{Алгебраическая функция. $f'(x)$.}
\end{figure}\\
\begin{figure}[h!]
\center{\includegraphics[width=0.6\textwidth]{Поли_вторая.png}}
\caption{Алгебраическая функция. $f''(x)$.}
\end{figure}\\
1. Из рисунка 4 видно,что $f(x)$ непрерывна на [a,b].\\
2. Из рисунка 5,6 видно, что $f'(x)$, $f''(x)$ закопостоянны на $[a,b]$.\\
3. Из рисунка 5 видно, что $f'(x)$ существует и отлична от 0 на [a,b].\\
4. Проверим условие Фурье:\\
$(-2 - 96 x^2 + 80 x^3)*(4x^5-8x^4-x^2+3x+5) = 1778, x= 2$\\
$(-2 - 96 x^2 + 80 x^3)*(4x^5-8x^4-x^2+3x+5) =54675, x = 2.5$\\
Все условия прошли проверку, метод секущих можно применить.\\
\section{Тестовый пример с расчетами}
\subsection{Метод половинного деления}
Возьмем $f(x)=x*cos(x)+3x^2-1$ на отрезке $[0.25,0.5]$. \\
Найдем корень с точностью $\epsilon=0.01$\\
\begin{enumerate}
\item $[0.25, 0.5]\ a=0.25,\ b = 0.5,\ c =0.375,\ f(a)*f(c)>0$
\item $[0.375, 0.5]\ a=0.375,\ b= 0.5,\ c =0.4375,\ f(a)*f(c)>0$
\item $[0.4375,0.5]\ a = 0.4375,\ b= 0.5,\ c = 0.46875 ,\ f(a)*f(c)<0$
\item $[0.4375,0.46875]\ a = 0.4375,\ b= 0.46875,\ c = 0.453125 ,\ f(a)*f(c)<0$\\
\end{enumerate}
$|0.4375-0.453125|<2*0.01$. Найден корень $x\approx 0.453125$с точонстью до $\epsilon$.\\
\subsection{Метод секущих}
Возьмем $f(x)=x*cos(x)+3x^2-1$ на отрезке $[0.5,0.75]$. \\
Найдем корень с точностью до $\epsilon = 0.001$\\
\begin{enumerate}
\item $prev = 0.75\ cur = 0.454\ next = 0.5 $
\item $prev = 0.454\ cur = 0.447\ next =0.75 $
\item $prev = 0.447\ cur = 0.4462\ next =0.454 $
\end{enumerate}
$(5.255 / (2 * 2.407) * |0.454 - 0.4462| * |0.4462 - 0.447| > \epsilon)$\\
Найден корень $x\approx 0.4462$ с точностью до $\epsilon = 0.001$\\
\section{Перечень контрольных тестов}
Для каждого из уравнений построим графики зависимостей 
\begin{enumerate}
    \item Зависимость абсолютной сходимости от номера итерации (разность точного корня и корня, полученного с помощью метода до точности $\epsilon = 10^{-12}$. Пошаговое приближение.\\
    \item Зависимость количества итерации от точности.
    \item Зависимость достигнутой точности от заданной. Достигнутая точность должна быть не больше заданного $\epsilon$.
    \item Зависимость количества итераций для достижения желаемой точности $\epsilon = 10^{-12}$ от выбора начального отрезка для МПД и двух начальных приближений для МС.
    \end{enumerate}
\section{Модульная структура программы}
\begin{itemize}
\item Константы, определенные через \#define, необходимые для корректной работы метода секущих. $f'(x)$ и $f''(x)$ соответственно.\\
\#m1\_TRANSC 2.40706\\
\# M2\_TRANSC 5.26296\\
\#define m1\_POLY 13.241\\
\#define M2\_POLY 52\\
\item double transcendental(double x); Функция, возвращающая значение трансцендентного уравнения в точке. На вход подаётся аргумент $x$.\\
\item double polynom(double x); Функция, возвращающая значение полинома в точке. На вход подаётся аргумент $x$.\\
\item typedef struct\\ 
\{\\
\	double root;\\
\	int iterations;\\
\} result\_t;\\
Структура, необходимая для удобного форматирования вывода результатов задачи.\\
\item result\_t bisection(double (f)(double), double a, double b, double eps);\\
Функция, реализующая непосредственно метод половинного деления.На вход подаётся указатель на функцию (нужно, чтобы работать с конкретным уравнением), левая и правая границы промежутка, желаемая точность.Возвращает структуру типа result\_t.\\
\item result\_t secant(double (f)(double), double x0, double x1, double eps, double m1, double M2)\\
Функция, реализующая непосредственно метод секущих. На вход подаётся указатель на функцию (нужно, чтобы работать с конкретным уравнением), левая и правая границы промежутка, желаемая точность.Возвращает структуру типа result\_t.\\
\item void bisection\_file(double (f)(double), double a, double b, double eps, FILE* file);\\
Функция, реализующая запись вычисленных промежуточных результатов МПД в отдельный файл. На вход принимает также указатель на функцию, границы промежутков, желаемую точность и поток файла, в который пишутся промежуточные значения. Ничего не возвращает.\\
\item void secant\_file(double (f)(double), double x0, double x1, double eps, double m1, double M2, FILE* file);\\
Функция, реализующая запись вычисленных промежуточных результатов метода секущих в отдельный файл. На вход принимает также указатель на функцию, границы промежутков, желаемую точность и поток файла, в который пишутся промежуточные значения. Ничего не возвращает.\\
\end{itemize}
\section{Численный анализ решения задачи}
\subsection{Абсолютная сходимость}
Исследуем зависимость абсолютной погрешности от номера итерации.
\begin{figure}[h!]
\center{\includegraphics[width=0.7\textwidth]{poly_abs_conv.png}}
\caption{Зависимость 1 для алгебраического уравнения}
\label{fig:Зависимость 1 для алгебраического уравнения}
\end{figure}\\
\begin{figure}[h!]
\center{\includegraphics[width=0.7\textwidth]{tr_abs_conv.png}}
\caption{Зависимость 1 для трансцендентного уравнения}
\label{fig:Зависимость 1 для трансцендентного уравнения}
\end{figure}\\
\newpage
Как видно из графиков 7 и 8, МПД сходится немонотонно, имеет линейную скорость сходимости, МС сходится монотонно и имеет сверхлинейную скорость сходимости.\\

\subsection{Быстродействие}
Исследуем зависимость количества итераций от точности.\\
Как видно из графиков 9, 10 МС на заданных отрезках сработал в ~9 раз быстрее, чем МПД для достижения желаемой точности.
\begin{figure}[h!]
\center{\includegraphics[width=0.7\textwidth]{poly_iter.png}}
\caption{Зависимость 2 для алгебраического уравнения}
\label{fig:Зависимость 2 для алгебраического уравнения}
\end{figure}\\
\begin{figure}[h!]
\center{\includegraphics[width=0.7\textwidth]{tr_iter.png}}
\caption{Зависимость 2 для трансцендентного уравнения}
\label{fig:Зависимость 2 для трансцендентного уравнения}
\end{figure}\\
\newpage
\subsection{Достижение желаемой точности}
Исследуем графики зависимости достигаемой точности.\\
Как видно из графиков 11, 12, ломаные не пересекают граничную линию $y=x$, что означает корректную работу МПД и МС. Точность на каждом шаге не превысила заданного $\epsilon$, что и ожидалось.\\
\begin{figure}[h!]
\center{\includegraphics[width=0.7\textwidth]{poly_acc.png}}
\caption{Зависимость 2 для алгебраического уравнения}
\label{fig:Зависимость 2 для алгебраического уравнения}
\end{figure}\\
\begin{figure}[h!]
\center{\includegraphics[width=0.7\textwidth]{tr_acc.png}}
\caption{Зависимость 2 для трансцендентного уравнения}
\label{fig:Зависимость 2 для трансцендентного уравнения}
\end{figure}\\
\newpage
\subsection{Зависимость итераций от начальных приближений}
Исследуем зависимость полученного количества итераций от начальных приближений. По, что МС практически не реагирует на изменение удалённости начальных приближений, в то время как количество итераций для МПД при заданной точности $\epsilon = 10^{-12}$ возрастает в среднем на 10.
\begin{figure}[h!]
\center{\includegraphics[width=0.7\textwidth]{poly_delta.png}}
\caption{Зависимость 2 для алгебраического уравнения}
\label{fig:Зависимость 2 для алгебраического уравнения}
\end{figure}\\
\begin{figure}[h!]
\center{\includegraphics[width=0.7\textwidth]{tr_delta.png}}
\caption{Зависимость 2 для трансцендентного уравнения}
\label{fig:Зависимость 2 для трансцендентного уравнения}
\end{figure}\\
\newpage
\subsection{Общие выводы}
В лабораторной работе я выполнил анализ метода половинного деления и метода секущих на примере алгебраического (полином) и трансцендентного уравнений.\\
В исследование входило: реализация методов, сравнение их сходимости, достигается ли с помощью них желаемая точность и за какое количество итераций, зависимость изменения количества итераций от начального промежутка от начальных приближений для МПД и МС соответственно. Подтверждено, что МПД сходится со скоростью, близко к линейной, но немонотонно, в то время как МС сходится сверхлинейно и на взятых мною примерах МС показал значительно лучше результаты, чем МПД. Скорее всего это обусловлено "удачным" выбором начальных приближений. МС может дать не такой значительное превосходство в скорости сходимости перед МПД, но сходиться будет всё равно быстрее.
\end{document}